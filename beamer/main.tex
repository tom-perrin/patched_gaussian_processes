

\documentclass[10pt, t,aspectratio=169, usepdftitle=true]{beamer} 

\usepackage{graphicx}
\usepackage{amsfonts}
\usepackage{amsmath}
\usepackage{amssymb}
\usepackage{tikz}
\usepackage{pifont}

\usepackage{natbib}

\newcommand*\circled[1]{\tikz[baseline=(char.base)]{
            \node[shape=circle,draw,inner sep=2pt] (char) {#1};}}

\def\withanim{0}

%%-----------------------------------------------------------------------
%% Preambule: Packages and command definitions
%%-----------------------------------------------------------------------


%% Logos
\newcommand{\logoHeader}{mines-vert.png}
\newcommand{\logoTitle}{\vspace{-5em}
	\includegraphics[width=15em]{mines.png}}

%% Preambule



%%-----------------------------------------------------------------------
%% Preambule: Packages and command definitions
%%-----------------------------------------------------------------------
\usefonttheme[onlymath]{serif}
\usetheme{Slides} 
\setbeamertemplate{navigation symbols}{} 



%%-----------------------------------------------------------------------
%% Packages
%%-----------------------------------------------------------------------


% Encoding
\usepackage[english]{babel}
\usepackage{amsmath,amsfonts,mathrsfs,bm,mathtools}
\usepackage{amssymb,eucal,dsfont}

% Multi-columns/rows
\usepackage{multicol,multirow,makecell}

% Bibliography
\usepackage{natbib}

% Adjustment boxes
\usepackage{adjustbox}

% To place text boxes
\usepackage[absolute,overlay]{textpos}
\usepackage[most]{tcolorbox}


\usepackage{graphicx,amssymb,amscd,mathdots}

\usepackage{enumerate}

% Hyperlinks
\usepackage{hyperref}
% Subfigures
\usepackage{subcaption}


\usepackage{array}
\newcolumntype{L}[1]{>{\raggedright\let\newline\\\arraybackslash\hspace{0pt}}m{#1}}
\newcolumntype{C}[1]{>{\centering\let\newline\\\arraybackslash\hspace{0pt}}m{#1}}
\newcolumntype{R}[1]{>{\raggedleft\let\newline\\\arraybackslash\hspace{0pt}}m{#1}}
\def\arraystretch{1.5}


% Center captions
\usepackage[justification=centering,skip=0.75ex,size=footnotesize]{caption}
\definecolor{colCaption}{gray}{0.25}
\captionsetup{textfont={color=colCaption}}
\captionsetup[figure]{labelformat=empty}

%\usepackage{verbatim}
% For animations
\usepackage{pdfpc-commands}

% For different numbering in appendix
% \usepackage{appendixnumberbeamer}

% TikZ
\usepackage{tikz}
\usetikzlibrary{arrows.meta}
\usetikzlibrary{positioning}
\usetikzlibrary{shapes, decorations.pathreplacing}

% Deal with spacing in commands
\usepackage{xspace}


% Color boxes
\usepackage{framed, tcolorbox}
\tcbset{colframe = white}
\tcbuselibrary{listings,breakable}

\usepackage{pdfpages}

%%-----------------------------------------------------------------------
%% Commands
%%-----------------------------------------------------------------------

% Operators
%\newcommand{\tr}[1]{#1^{T}}
\newcommand{\tr}[1]{{}^t{#1}}
\newcommand{\diag}{\mathrm{diag}}
\newcommand{\Diag}{\mathrm{Diag}}
\DeclareMathOperator*{\argmin}{arg\,min}
\DeclareMathOperator*{\argmax}{arg\,max}
\newcommand{\ft}{\mathscr{F}}
\newcommand{\cont}{\mathscr{C}}
\newcommand{\neig}{\mathrm{ne}}
\newcommand{\spn}[1]{\mathrm{span}\left\lbrace #1 \right\rbrace}


%% Random variables
\newcommand{\proba}[1]{\mathbb{P}\left[#1\right]}
\newcommand{\e}{\mathbb{E}}
\newcommand{\cov}{\mathrm{Cov}}
\newcommand{\var}{\mathrm{Var}}

%% Sets of numbers
\newcommand{\C}{\mathbb{C}}
\newcommand{\R}{\mathbb{R}}
\newcommand{\N}{\mathbb{N}}

% Quotation marks
\newcommand{\q}[1]{``#1''}

% Text in color
\newcommand{\red}[1]{\textcolor{red}{#1}}
\newcommand{\tc}[2]{\textcolor{#1}{#2}}

% Sets of integres
\newcommand{\bi}{[\![}
\newcommand{\ei}{]\!]}

% Display stile in inline equation
\newcommand{\ds}{\displaystyle}

% Additional commands
\newcommand{\dv}{\mathrm{div}}
\newcommand{\mc}[1]{\mathcal{#1}}
\newcommand{\K}{\mathrm{K}}
\newcommand{\sk}{\mathrm{SK}}
\newcommand{\mk}{\mathrm{MK}}
\newcommand{\ok}{\mathrm{OK}}
\newcommand{\fk}{\mathrm{FK}}
\newcommand{\ck}{\mathrm{CoK}}
\newcommand{\cD}{\mathcal{D}}
\newcommand{\cZ}{\mathcal{Z}}
\newcommand{\cY}{\mathcal{Y}}
\newcommand{\dd}{\mathrm{d}}
\newcommand{\tp}[1]{{{}^t{#1}}}



%%-----------------------------------------------------------------------
%% Environments
%%-----------------------------------------------------------------------


% Center without space
\newenvironment{nscenter}
{\parskip=0pt\par\nopagebreak\centering}
{\par\noindent\ignorespacesafterend}

% Algorithms
\newenvironment<>{alg}[1]{
	\begin{nscenter}
		\begin{minipage}{\linewidth}
			\begin{mdframed}[backgroundcolor=mygreylight,linewidth=0pt]
				\begin{algorithm}[H]}{
				\end{algorithm}
			\end{mdframed}
		\end{minipage}
	\end{nscenter}
}

		
\renewenvironment<>{greyblock}[1]
{\begin{tcolorbox}[title=#1,arc=0pt,colback=greyBlockColor,colbacktitle=greyTitleBlockColor,boxrule=0pt,coltext=black,coltitle=white]}
	{\end{tcolorbox}}



%%-----------------------------------------------------------------------
%% Other
%%-----------------------------------------------------------------------

% Add space between items in itemize
\let\OLDitemize\itemize
\renewcommand\itemize{\OLDitemize\addtolength{\itemsep}{0.5em}}


% Define colors
\definecolor{c1}{RGB}{56, 103, 161}
\definecolor{c2}{RGB}{245, 87, 66}
\definecolor{c3}{RGB}{245, 87, 66}
\definecolor{c4}{RGB}{60, 138, 51}
\definecolor{c5}{RGB}{0, 0, 0}
\definecolor{mygreen}{RGB}{63,92,44}


\def\boxitem#1{\vspace{0pt}\setbox0=\vbox{#1}{\centering\makebox[0pt]{%
			\fboxrule=2pt\color{gray}\fbox{\hspace{\leftmargini}\color{black}\box0}}\vspace{3ex}\par}}






% Grey out items
\setbeamercovered{still covered={\opaqueness<1->{0}},again covered={\opaqueness<1->{60}}}


%%-----------------------------------------------------------------------
%% Presentation characteristics
%%-----------------------------------------------------------------------

% Title 
\title{Patched Gaussian Processes}


% Authors
\author[T. Perrin, L. Roux \& T. Lambert]{\textbf{Tom Perrin, Leo Roux \& Thibault Lambert}}

\institute{Option Géostatistique, Mines Paris - PSL 
    \vspace{1em}\\
	\texttt{tom.perrin@etu.minesparis.psl.eu\\
            leo.roux@etu.minesparis.psl.eu\\
            thibault.lambert@etu.minesparis.psl.eu}\\
	\vspace{1em}
	%{\textit{Joint work with N. Desassis (Mines Paris), A. Lang (Chalmers), L. Clarotto (AgroParisTech)}}
}

% Event
%\date{SAMA @ Aussois \\ September 16th, 2024}
\date{}


%%-----------------------------------------------------------------------
%%----------------------------------------------------------------------
%%-----------------------------------------------------------------------
%%----------------------------------------------------------------------


\begin{document}

	\nocite{*}	% adds references with citing

	% Equations
	\setlength{\abovedisplayskip}{0.75ex}
	\setlength{\belowdisplayskip}{0.75ex}
	
	%%----------------------------------------------------------------------
	
	\begin{frame}[plain,c]
		\titlepage
	\end{frame}
	
	%%----------------------------------------------------------------------
	
	
	
	%%-------------------------- Outline frame before each section
	\AtBeginSection[]{
		\begin{frame}
			\frametitle{Table of contents}
			\tableofcontents[currentsection]
		\end{frame} 
	}
	
	\AtBeginSubsection[]{
		\begin{frame}
			\frametitle{Table of contents}
			\tableofcontents[currentsubsection]
		\end{frame} 
	}
	%%--------------------------
	
	
	
	%%----------------------------------------------------------------------
	%%----------------------------------------------------------------------
	


	\section{Prerequisites}
	
	\begin{frame}{Gaussian Process}
		\textbf{Gaussian Process :} Random surface $W(s)$ over a domain $\mathcal{D}$ such that :
		$$ \forall \{s_1,\dots,s_n\} \in \mathcal{D}^n, W := (W(s_1),\dots,W(s_n)) \sim \mathcal{N}(\mu, C).$$

		\vspace{0.5cm}

		\begin{itemize}
    			\item \textbf{Mean function :} $\mu(s) := \mathbb{E}[W(s)] \longrightarrow \mu := [\mu(s_1), \dots, \mu(s_N)]^T$
			 \item \textbf{Covariance function :} $C(s,s') := \mathbb{C}\mathrm{ov}[W(s),W(s')] \longrightarrow C := [C(s_i, s_j)]_{1\leqslant i,j \leqslant N}.$
		\end{itemize}

		\vspace{1cm}

		\textbf{Law of $\boldsymbol{W}$ :}
		
		$$p(W) = \frac{1}{\sqrt{(2\pi)^d\text{det}(C)}}\exp \left(-\frac{1}{2} (W - \mu)^T C ^{-1} (W - \mu) \right).$$

	\end{frame}

	
	\begin{frame}{Kriging}
		\textbf{Observations :} $\mathcal{D} = \{(x_i,y_i) \mid i \in \{1,\dots,N\}\}.$
		\begin{center}
    			\textbf{Locations $\boldsymbol{x} := [x_1,\dots,x_N]^T$} $\longrightarrow$ \textbf{Responses $\boldsymbol{y} := [y_1,\dots,y_N]^T$}.
		\end{center}

		\vspace{0.5cm}
	
		\textbf{Kriging :} Stochastic predictions based on these observations.
		\begin{itemize}
    			\item \textbf{Reponses} $\boldsymbol{y}$ seen as realizations of a \textbf{gaussian process} $\boldsymbol{f}$ with noise :
    			\begin{center}
        				$ y_i = f(x_i) + \varepsilon_i,$ where $\varepsilon_i \overset{\text{i.i.d.}}{\sim} \mathcal{N}(0,\sigma^2).$
			 \end{center}
			\item \textbf{Prediction} $\boldsymbol{f_* := f(x^*)}$ at new location $x^*$ $\longrightarrow$ \textbf{Joint law of} $\boldsymbol{(f_*,y)}$:
   			 \begin{center}
        				$ p(f_*,y) \sim \mathcal{N} \left( 0, 
       			 	\begin{bmatrix}
            				c_{**} & c_{*x} \\
            				c_{x*} & \sigma^2I+C_{xx}
       				 \end{bmatrix} \right).$
    			\end{center}
    			\scriptsize where $c_{**} = C(x^*,x^*), c_{x*}=[C(x_1,x^*),\dots,C(x_N,x^*)]^T, C_{xx} = [C(x_i,x_j)]_{1\leqslant i,j \leqslant N}.$
		\end{itemize}
   			 
	\end{frame}


	\begin{frame}{Kriging}
		\textbf{Predictive distribution :}
    		$$p(f_* \mid y) \sim \mathcal{N}(
			\underbrace{c_{x*}^T(\sigma^2I+C_{xx})^{-1}y}_{\text{predictive mean}},
        			\underbrace{c_{**} - c_{x*}^T(\sigma^2I+C_{xx})^{-1}c_{x*}}_{\text{predictive variance}})$$

		\vspace{0.5cm}

		\begin{itemize}
   			 \item \textbf{Predicted value :} $f_*(x^*) \approx c_{x*}^T(\sigma^2I+C_{xx})^{-1}y.$
  		 	 \item \textbf{Uncertainty :} $c_{**} - c_{x*}^T(\sigma^2I+C_{xx})^{-1}c_{x*}.$
		\end{itemize} 
	\end{frame}

	
	\begin{frame}{Bayesian Approach}
		\textbf{Hierarchical model :} 
		$$y_i = X(x_i) \cdot \beta + f_{\theta}(x_i) + \varepsilon_i.$$

		\vspace{0.5cm}

		\begin{itemize}
      			 \item \textbf{Regression term $\boldsymbol{X(x_i) \cdot \beta}$ :} Captures effects of known explanatory variables $X$.
       			 \item \textbf{Spatial effect $\boldsymbol{f_{\theta}(x_i)}$ :} Realization of a gaussian process $f_\theta \sim \mathcal{N}(0, C_\theta).$
       			 \item \textbf{Noise $\boldsymbol{\varepsilon_i}$ :} $\varepsilon_i \overset{iid}{\sim} \mathcal{N}(0, \tau^2).$
    		\end{itemize}
	\end{frame}

	
	\begin{frame}{Bayesian Approach}
		\textbf{Bayesian Inference :} Estimation of the \textbf{hidden state} $\boldsymbol{f_\theta}$ and \textbf{parameters} $\boldsymbol{\phi} := \{\beta, \theta, \tau^2\}$.
		
		\vspace{0.5cm}

		\begin{itemize}
			\item \textbf{Prior distributions :} We assume prior knowledge $p(\beta)$, $p(\theta)$ and $p(\tau^2)$.
			\item \textbf{Likelihood :} $p(y \mid f_\theta, \beta, \tau^2) = \mathcal{N}(X \beta + f_\theta, \tau^2 I).$
			\item \textbf{Latent GP prior :} $p(f_\theta \mid \theta) = \mathcal{N}(0, C_\theta).$
		\end{itemize}

		\vspace{1cm}

		\textbf{Joint posterior distribution :}
		$$p(f_\theta, \beta, \theta, \tau^2 \mid y) \propto \underbrace{p(y \mid f_\theta, \beta, \tau^2)}_{\text{data likelihood}} \times \underbrace{p(f_\theta \mid \theta)}_{\text{spatial link}} \times \underbrace{p(\beta)p(\theta)p(\tau^2)}_{\text{parameter priors}}.$$
	\end{frame}
	
	%%----------------------------------------------------------------------
	%%----------------------------------------------------------------------
	


\section{The "Big $N$" Problem and Beyond}
	
	\begin{frame}{Computational Costs}
		Both the predictive mean and variance require \textbf{solving linear systems} involving $\Sigma = \sigma^2 I + C_{xx}$.

		\vspace{0.5cm}

		\begin{itemize}
			\item \textbf{Time complexity :} Matrix inversion $\longrightarrow \boldsymbol{\mathcal{O}(N^3)}.$
			\item \textbf{Space complexity :} Storing covariance matrix $\longrightarrow \boldsymbol{\mathcal{O}(N^2)}.$
		\end{itemize}

		\vspace{1cm}

		\textbf{Solutions :}
		\begin{itemize}
			\item Acting on the \textbf{covariance matrix} $\boldsymbol{\Sigma}$ : compact support, covariance tapering, markovian models $\longrightarrow$ \textbf{Sparse matrix}.
			\item Acting on the \textbf{amount of observations} $\boldsymbol{N}$...
		\end{itemize}
	\end{frame}


   	 \begin{frame}{Independant Local Kriging}
		\textbf{Idea :} Split the observations $\mathcal{D}$ into $K$ subsets $\mathcal{D}_1,\dots,\mathcal{D}_K$, with $\mathcal{D}_k := \left \{ (x_i, y_i) \mid x_i \in \Omega_k \right \}.$
		\begin{center}
			Region $\Omega_k \longrightarrow$ Local gaussian process $f_k \longrightarrow$ Associated covariance function $C_k(\cdot, \cdot)$.
		\end{center}
		
		\vspace{0.5cm}

		\begin{itemize}
			\item \textbf{Stationary process :} $C_k(\cdot,\cdot) = C(\cdot,\cdot)$.
			\item Otherwise : Different covariance functions $C_k$.
		\end{itemize}
		$$y_{k,i} = f_k(x_i) + \varepsilon_{k,i}.$$

		\vspace{0.5cm}

		Two questions :
		\begin{itemize}
			\item How to split the observations efficiently ?
			\item How to deal with shared boundaries $\Gamma_{k, \ell} := \overline{\Omega}_k \cap \overline{\Omega}_{\ell}$ ?
		\end{itemize}
	\end{frame}


	\begin{frame}{Splitting the observations}
		\begin{columns}[T]
			\column{0.5 \textwidth}
				\textbf{Grid partitioning :} Splitting the data using an uniform grid to cover the space.
				\vspace{0.2cm}
				\begin{itemize}
					\item[\circled{$+$}] Very easy implementation.
					\item[\circled{$-$}] High density variance between regions.
				\end{itemize}
				
				\vspace{1cm}				

				\textbf{PCA partitioning :} Splitting the data based on principal component projections values.
				\vspace{0.2cm}
				\begin{itemize}
					\item[\circled{$+$}] Balanced number of points ($N_k \approx \text{cst}$).
					\item[\circled{$-$}] Complex boundary definition.
				\end{itemize}
			\column{0.4 \textwidth}
         				 \includegraphics[width=0.67\linewidth]{figures/partitioning_grid.png}
           			 \par\footnotesize \hspace{0.9cm} Grid partitioning\par
            
            			\vspace{0.5cm}
            
            			\includegraphics[width=0.67\linewidth]{figures/partitioning_pca.png}
           			\par\footnotesize \hspace{0.9cm} PCA partitioning\par

		\end{columns}
	\end{frame}
	

	\begin{frame}{Border discontinuity}
		\begin{columns}[T]
			\column{0.4 \textwidth}	
				\textbf{Issue :}
				\begin{center}
					$f_k(x) \neq f_\ell(x)$ at frontier $\Gamma_{k,\ell}.$
				\end{center}
				
				\vspace{0.5cm}

				\textbf{Solution :}
				\begin{center}
					Set $\delta_{k,\ell} := f_k - f_\ell = 0$ at $\Gamma_{k,\ell}.$
				\end{center}
				
				\vspace{0.5cm}
				
				In practice : pseudo-observations $\delta_{k,\ell}(x) = 0$ at the frontiers. \\
				\hspace{0.2cm} $\mathbb{E} \left[ f_*^{(k)} \mid y \right] \longrightarrow \mathbb{E} \left[ f_*^{(k)} \mid y, \delta = 0 \right].$ \\
				\hspace{0.2cm} $\mathbb{V} \left[ f_*^{(k)} \mid y \right] \longrightarrow \mathbb{V} \left[ f_*^{(k)} \mid y, \delta = 0 \right].$

			\column{0.5 \textwidth}
				\begin{figure}
					\centering
					\includegraphics[width=\linewidth]{figures/border_discontinuity.png}
           				\par\footnotesize Potential discontinuity at frontier \par
				\end{figure}
		\end{columns}

	\end{frame}

	%%----------------------------------------------------------------------
	%%----------------------------------------------------------------------
	
	
	\section{Methods to Tackle the Issue}

    \subsection{Patchwork Kriging (Park and Aley)}
	
	\begin{frame}{Pseudo-observations}
		For two neighbor regions $\Omega_k, \Omega_\ell$, place $B$ observations on the border $\Gamma_{k,\ell} :$
		\begin{itemize}
			\item \textbf{Pseudo-locations :} $x^{(k,\ell)} := \left(x^{(k,\ell)}_1,\dots,x^{(k,\ell)}_B\right)^T.$
			\item \textbf{Pseudo-values :} $\delta_{k,\ell} := \left(\delta_{k,\ell}(x^{(k,\ell)}_1),\dots,\delta_{k,\ell}(x^{(k,\ell)}_B)\right)^T.$
		\end{itemize}

		\vspace{0.5cm}

		\begin{itemize}
			\item[\ding{228}] \textbf{Observations :} $y := \left(y_1^T,\dots,y_K^T\right)^T.$
			\item[\ding{228}] \textbf{Pseudo-observations :} $\delta := \left(\delta_{1,1}^T,\dots,\delta_{1,K}^T,\dots,\delta_{K,K}^T\right)^T.$
		\end{itemize}
	\end{frame}

   	\begin{frame}{Patchwork Kriging}
		Predict $f_*^{(k)} := f_k(x^*)$ at $x^* \in \Omega_k \longrightarrow$ \textbf{Joint law of} $\boldsymbol{(f_*^{(k)}, y, \delta)} :$

		\vspace{0.5cm}		

		\begin{equation*}
    			\begin{bmatrix}
       				f^{(k)}_* \\
        				y \\
        				\delta
			\end{bmatrix} \sim \mathcal{N} \left(
   			\begin{bmatrix}
        				0 \\
        				0 \\
        				0
    			\end{bmatrix},
    			\begin{bmatrix}
       				c_{**} & c^{(k)}_{*\mathcal{D}} & c^{(k)}_{*\delta} \\
        				c^{(k)}_{\mathcal{D}*} & C_{\mathcal{D}\mathcal{D}} & C_{\mathcal{D}\delta} \\
        				c^{(k)}_{\delta*} & C_{\delta\mathcal{D}} & C_{\delta\delta}
    			\end{bmatrix} \right).
    		\end{equation*}

		\vspace{0.5cm}		

		\begin{itemize}
			\item On the border between regions $k$ and $l$ $\delta_{k,l} = f_k - f_l$
			\item Under the asumption $f_k$ and $f_l$ are independent gaussian processes, $\delta_{k,l}$ is one too
			\item Specifically, we can easily compute its covariance with $f_j(x)$, for a certain $x \in\mathcal{D}_j$
		\end{itemize}
		$$
			\mathbb{C}\text{ov}(\delta_{k,l}(x_1) , f_j(x_2)) = \mathbb{C}\text{ov}(f_k(x_1) , f_j(x_2)) - \mathbb{C}\text{ov}(f_l(x_1) , f_j(x_2))
		$$
	\end{frame}

	\begin{frame}{Patchwork Kriging}
		Definitions of the covariance blocks :
		\small
		\begin{equation*}
			\begin{split}
			c_{**} & := \mathbb{C}\text{ov}(f_*^{(k)}, f_*^{(k)}), \\
			c_{*\mathcal{D}}^{(k)} & := \left(\mathbb{C}\text{ov}(f_*^{(k)}, y_1),\dots,\mathbb{C}\text{ov}(f_*^{(k)}, y_K) \right), \\ 
			c_{*\delta}^{(k)} & := \left(\mathbb{C}\text{ov}(f_*^{(k)}, \delta_{1,1}),\dots,\mathbb{C}\text{ov}(f_*^{(k)}, \delta_{K,K}) \right),
			\end{split}
		\end{equation*}
		\begin{equation*}
			\begin{split}
			C_{\mathcal{D}\mathcal{D}} & := 
			\begin{bmatrix}
				\mathbb{C}\text{ov}(y_1, y_1) & \dots & \mathbb{C}\text{ov}(y_1, y_K) \\
				\ldots & \ddots & \ldots \\
				\mathbb{C}\text{ov}(y_K, y_1) & \dots & \mathbb{C}\text{ov}(y_K, y_K)
			\end{bmatrix}, \hspace{0.2cm}
			C_{\mathcal{D}\delta} :=
			\begin{bmatrix}
				\mathbb{C}\text{ov}(y_1, \delta_{1,1}) & \dots & \mathbb{C}\text{ov}(y_1, \delta_{K,K}) \\
				\ldots & \ddots & \ldots \\
				\mathbb{C}\text{ov}(y_K,  \delta_{1,1}) & \dots & \mathbb{C}\text{ov}(y_K,  \delta_{K,K})
			\end{bmatrix} \hspace{0.2cm} \text{and}  \\
			C_{\delta\delta} & := 
			\begin{bmatrix}
				\mathbb{C}\text{ov}(\delta_{1,1}, \delta_{1,1}) & \dots & \mathbb{C}\text{ov}(\delta_{1,1}, \delta_{K,K}) \\
				\ldots & \ddots & \ldots \\
				\mathbb{C}\text{ov}(\delta_{K,K},  \delta_{1,1}) & \dots & \mathbb{C}\text{ov}(\delta_{K,K},  \delta_{K,K})
			\end{bmatrix}.
			\end{split}
		\end{equation*}
	\end{frame}

  	 \begin{frame}{Patchwork Kriging}
		\textbf{Predictive mean :} \vspace{0.2cm}
		$$ \mathbb{E} \left[ f_*^{(k)} \mid y, \delta = 0 \right] = \left( c_{*\mathcal{D}}^{(k)} - c_{*\delta}^{(k)} C_{\delta\delta}^{-1} C_{\delta\mathcal{D}} \right) \left(  C_{\mathcal{D}\mathcal{D}} - C_{\mathcal{D}\delta} C_{\delta\delta}^{-1} C_{\delta\mathcal{D}} \right)^{-1} y.$$

		\vspace{0.2cm}

		\textbf{Predictive variance :} \vspace{0.2cm}
		\begin{equation*}
			\begin{split}
			\mathbb{V} \left[ f_*^{(k)} \mid y, \delta = 0 \right] = & c_{**} - c_{*\delta}^{(k)}C_{\delta\delta}^{-1}c_{\delta*}^{(k)} \\
			& - \left( c_{*\mathcal{D}}^{(k)} - c_{*\delta}^{(k)} C_{\delta\delta}^{-1} C_{\delta\mathcal{D}} \right) \left(  C_{\mathcal{D}\mathcal{D}} - C_{\mathcal{D}\delta} C_{\delta\delta}^{-1} C_{\delta\mathcal{D}} \right)^{-1} \left( c_{\mathcal{D}*}^{(k)} - c_{\delta*}^{(k)} C_{\delta\delta}^{-1} C_{\mathcal{D}\delta} \right) 
			\end{split}
		\end{equation*}

		\vspace{0.5cm}

		Noting $Q:=\left(  C_{\mathcal{D}\mathcal{D}} - C_{\mathcal{D}\delta} C_{\delta\delta}^{-1} C_{\delta\mathcal{D}} \right)^{-1}, v:=L^{-1}C_{\delta\mathcal{D}}$ and $w_*:=L^{-1}c_{\delta*}^{(k)}$ (where $C_{\delta\delta} = LL^T$),

		\begin{equation*}
			\boxed{
			\begin{split}
			\mathbb{E} \left[ f_*^{(k)} \mid y, \delta = 0 \right] & = (c_{*\mathcal{D}}^{(k)} - w_*^Tv) Q y.\\
			\mathbb{V} \left[ f_*^{(k)} \mid y, \delta = 0 \right] & = c_{**} - w_*^T w_* - (c_{*\mathcal{D}}^{(k)} - w_*^Tv) Q (c_{*\mathcal{D}}^{(k)} - w_*^Tv)^T.
			\end{split}
			}
		\end{equation*}
	\end{frame}

	\begin{frame}{Proof}
		\scriptsize For two gaussian vectors $\bold{x}_1 \sim \mathcal{N}(\mu_1, \Sigma_1)$ and $\bold{x}_2 \sim \mathcal{N}(\mu_2, \Sigma_2)$, the conditional distribution $p(\bold{x}_1 \mid \bold{x}_2)$ is gaussian and verifies :
		$$
		\begin{aligned}
			\mathbb{E} \left[\mathbf{x}_1 \mid \mathbf{x}_2 \right] & = \mu_1 + \Sigma_{12} \Sigma_{22}^{-1} (\bold{x}_2 - \mu_2). \\
			\mathbb{V} \left[\mathbf{x}_1 \mid \mathbf{x}_2 \right] & = \Sigma_{11} - \Sigma_{12} \Sigma_{22}^{-1} \Sigma_{21}.
		\end{aligned}
		$$

		\vspace{0.5cm}

		Applying this to $\mathbf{x}_1 = f_*^{(k)}$ and $\mathbf{x}_2 = (y, \delta)$ gives :
		$$
		\mathbb{E}\left[f_*^{(k)} \mid y, \delta \right] = \left[c_{* \mathcal{D}}^{(k)}, c_{*\delta}^{(k)}\right] \left[
		\begin{array}{ll}
			C_{\mathcal{D} \mathcal{D}} & C_{\mathcal{D}\delta} \\
			C_{\delta\mathcal{D}} & C_{\delta\delta}
		\end{array}\right]^{-1}\left[
		\begin{array}{l}
			{y} \\
			{\delta}
		\end{array}\right]
		$$
		\vspace{0.2cm}
		and
		\vspace{0.2cm}
		$$
		\mathbb{V}\left[f_*^{(k)} \mid y, \delta \right] = c_{**} - \left[c_{* \mathcal{D}}^{(k)}, c_{*\delta}^{(k)}\right] \left[
		\begin{array}{ll}
			C_{\mathcal{D} \mathcal{D}} & C_{\mathcal{D}\delta} \\
			C_{\delta\mathcal{D}} & C_{\delta\delta}
		\end{array}\right]^{-1}\left[
		\begin{array}{l}
			c_{\mathcal{D}*}^{(k)} \\
			c_{\delta*}^{(k)}
		\end{array}\right].
		$$
	\end{frame}

	\begin{frame}{Proof}
		\scriptsize
		For invertible matrix $A, B, D,$ the following inversion formula holds true :
		$$
		\left[
		\begin{array}{cc}
			A & B \\
			B^T & D
		\end{array}\right]^{-1} = \left[
		\begin{array}{cc}
			\left(A-B D^{-1} B^T\right)^{-1} & -\left(A-B D^{-1} B^T\right)^{-1} B D^{-1} \\
			- D^{-1} B^T\left(A-B D^{-1} B^T\right)^{-1} & \left(D-B^T A^{-1} B\right)^{-1}
		\end{array}\right].
		$$

		\vspace{0.5cm}

		Applying this to $A=C_{\mathcal{D}\mathcal{D}}, B=C_{\mathcal{D}\delta}, D=C_{\delta\delta}$ gives, for the predictive mean :
		$$
		\begin{aligned}
			\mathbb{E}\left[f_*^{(k)} \mid y, \delta \right] = &  \left[
			\begin{array}{l}
				c_{* \mathcal{D}}^{(k)} \\
				c_{*\delta}^{(k)}
			\end{array} \right]^T \left[
			\begin{array}{cc}
				 \left(C_{\mathcal{D} \mathcal{D}}-C_{\mathcal{D} \delta} C_{\delta\delta}^{-1} C_{\delta\mathcal{D}} \right)^{-1} & -\left(C_{\mathcal{D} \mathcal{D}}-C_{\mathcal{D} \delta} C_{\delta\delta}^{-1} C_{\delta\mathcal{D}} \right)^{-1} C_{\mathcal{D}\delta} C_{\delta\delta}^{-1} \\
				- C_{\delta\delta}^{-1} C_{\delta\mathcal{D}} \left(C_{\mathcal{D} \mathcal{D}}-C_{\mathcal{D} \delta} C_{\delta\delta}^{-1} C_{\delta\mathcal{D}} \right)^{-1} & \left(C_{\delta\delta}-C_{\delta\mathcal{D}} C_{\mathcal{D} \mathcal{D}}^{-1} C_{\mathcal{D} \delta}\right)^{-1}
			\end{array}\right] \left[
			\begin{array}{l}
				{y} \\
				{\delta}
			\end{array}\right]
			\\
			= & c_{* \mathcal{D}}^{(k)}\left(C_{\mathcal{D} \mathcal{D}}-C_{\mathcal{D} \delta} C_{\delta\delta}^{-1} C_{\delta\mathcal{D}} \right)^{-1} y-c_{*\delta}^{(k)} C_{\delta\delta}^{-1} C_{\mathcal{D} \delta}^T\left(C_{\mathcal{D} \mathcal{D}}-C_{\mathcal{D} \delta} C_{\delta\delta}^{-1} C_{\delta\mathcal{D}} \right)^{-1} y \\
			& - c_{* \mathcal{D}}^{(k)}\left(C_{\mathcal{D} \mathcal{D}}-C_{\mathcal{D} \delta} C_{\delta\delta}^{-1} C_{\delta\mathcal{D}} \right)^{-1} C_{\mathcal{D} \delta} C_{\delta\delta}^{-1} \delta+c_{* \delta}^{(k)}\left(C_{\delta\delta}-C_{\delta\mathcal{D}} C_{\mathcal{D} \mathcal{D}}^{-1} C_{\mathcal{D} \delta}\right)^{-1} \delta.
		\end{aligned}
		$$

		\vspace{0.5cm}

		Taking $\delta = 0$ finally leads to :
 $$ \mathbb{E} \left[ f_*^{(k)} \mid y, \delta = 0 \right] = \left( c_{*\mathcal{D}}^{(k)} - c_{*\delta}^{(k)} C_{\delta\delta}^{-1} C_{\delta\mathcal{D}} \right) \left(  C_{\mathcal{D}\mathcal{D}} - C_{\mathcal{D}\delta} C_{\delta\delta}^{-1} C_{\delta\mathcal{D}} \right)^{-1} y.$$
	\end{frame}

	\begin{frame}{Proof}
		\scriptsize
		Concerning the predictive variance, the inversion formula also gives :
		\vspace{0.2cm}
		\begin{equation*}
			\begin{aligned}
			& \hspace{0.1cm} \mathbb{V}\left[f_*^{(k)} \mid y, \delta \right] \\
			= & \hspace{0.1cm}  c_{**} - \left[
			\begin{array}{l}
				c_{* \mathcal{D}}^{(k)} \\
				c_{*\delta}^{(k)}
			\end{array} \right]^T \left[
			\begin{array}{cc}
				 \left(C_{\mathcal{D} \mathcal{D}}-C_{\mathcal{D} \delta} C_{\delta\delta}^{-1} C_{\delta\mathcal{D}} \right)^{-1} & -\left(C_{\mathcal{D} \mathcal{D}}-C_{\mathcal{D} \delta} C_{\delta\delta}^{-1} C_{\delta\mathcal{D}} \right)^{-1} C_{\mathcal{D}\delta} C_{\delta\delta}^{-1} \\
				- C_{\delta\delta}^{-1} C_{\delta\mathcal{D}} \left(C_{\mathcal{D} \mathcal{D}}-C_{\mathcal{D} \delta} C_{\delta\delta}^{-1} C_{\delta\mathcal{D}} \right)^{-1} & \left(C_{\delta\delta}-C_{\delta\mathcal{D}} C_{\mathcal{D} \mathcal{D}}^{-1} C_{\mathcal{D} \delta}\right)^{-1}
			\end{array}\right] \left[
			\begin{array}{l}
				c_{\mathcal{D}*}^{(k)} \\
				c_{\delta*}^{(k)}
			\end{array}\right] \\
			= & \hspace{0.1cm} c_{**} - c_{*\mathcal{D}}^{(k)} \left(C_{\mathcal{D} \mathcal{D}}-C_{\mathcal{D} \delta} C_{\delta\delta}^{-1} C_{\delta\mathcal{D}} \right)^{-1} c_{\mathcal{D}*}^{(k)} + c_{*\mathcal{D}}^{(k)} \left(C_{\mathcal{D} \mathcal{D}}-C_{\mathcal{D} \delta} C_{\delta\delta}^{-1} C_{\delta\mathcal{D}} \right)^{-1} C_{\mathcal{D}\delta} C_{\delta\delta}^{-1} c_{\delta*}^{(k)} \\
			& \hspace{0.1cm} + c_{*\delta}^{(k)} C_{\delta\delta}^{-1} C_{\delta\mathcal{D}}  \left(C_{\mathcal{D} \mathcal{D}}-C_{\mathcal{D} \delta} C_{\delta\delta}^{-1} C_{\delta\mathcal{D}} \right)^{-1} c_{\mathcal{D}*}^{(k)} - c_{*\delta}^{(k)} \left( C_{\delta\delta} - C_{\delta\mathcal{D}} C_{\mathcal{D}\mathcal{D}}^{-1} C_{\mathcal{D}\delta} \right)^{-1} c_{\delta*}^{(k)}.
			\end{aligned}
		\end{equation*}

		\vspace{0.5cm}

		This finally yields to :

		$$
		\mathbb{V}\left[f_*^{(k)} \mid y, \delta = 0 \right] = c_{**} - c_{*\delta}^{(k)} C_{\delta\delta}^{-1} c_{\delta*}^{(k)} - \left( c_{*\mathcal{D}}^{(k)} - c_{*\delta}^{(k)} C_{\delta\delta}^{-1} C_{\delta\mathcal{D}} \right) \left(C_{\mathcal{D} \mathcal{D}}-C_{\mathcal{D} \delta} C_{\delta\delta}^{-1} C_{\delta\mathcal{D}} \right)^{-1} \left( c_{\mathcal{D}*}^{(k)} - C_{\mathcal{D}\delta} C_{\delta\delta}^{-1} c_{\delta*}^{(k)} \right).
		$$
	\end{frame}

	\begin{frame}{Choice of the parameters}
		\textbf{Covariance function :}
		\begin{itemize}
			\item Different for each region $\longrightarrow c_k(\cdot,\cdot)$ for $k\in\{1,\dots,K\}.$ 
			\item Obtained by minimizing the negative log-likelihood :
		\end{itemize}
		$$
		\begin{aligned}
		NL(\theta) &  :=  - \log p(y,\delta=0 \mid \theta) \\
		& = \frac{N}{2} \log(2\pi) + \frac{1}{2} \log
		\begin{vmatrix}
			C_{\mathcal{D}\mathcal{D}} & C_{\mathcal{D}\delta} \\
			C_{\delta\mathcal{D}} & C_{\delta\delta}
		\end{vmatrix}
		+ \frac{1}{2} \left[ 
		\begin{array}{l}
			y \\
			0
		\end{array}
		\right]^T
		\begin{bmatrix}
			C_{\mathcal{D}\mathcal{D}} & C_{\mathcal{D}\delta} \\
			C_{\delta\mathcal{D}} & C_{\delta\delta}
		\end{bmatrix}^{-1} \left[
		\begin{array}{l}
			y \\
			0
		\end{array}
		\right].
		\end{aligned}
		$$
	
		\vspace{0.5cm}

		\textbf{Hyperparameters :}
		\begin{itemize}
			\item Number of regions $\boldsymbol{K}=2^D$ (where $D$ is the depth of the PCA partitioning).
			\begin{itemize}
				\item[\ding{228}] $K \nearrow \implies$ Less accurate predictions.
	    		\end{itemize}
			\item Number of pseudo-observations $\boldsymbol{B}$ per frontier.
			\begin{itemize}
				\item[\ding{228}] $B \nearrow \implies$ More accurate predictions.
			\end{itemize}
		\end{itemize}
	\end{frame}
		

	%%----------------------------------------------------------------------

    \subsection{Nearest-Neighbors Gaussian Processes (Datta et al.)}
	
	\begin{frame}{Intuitive Idea}
		\begin{itemize}
			\item It seems logical to assume that far away observations have little impact on the Kriging.
			\item Instead Kriging the whole dataset,  we are Kriging each $x^{*}$ only with the nearest neighbors.
		\end{itemize}
	\end{frame}

  	 \begin{frame}{Theoretical Definition}
		\begin{itemize}
			\item Item
			\item Item
		\end{itemize}
	\end{frame}

   	\begin{frame}{Proper Solution}
		\begin{itemize}
			\item Item
			\item Item
		\end{itemize}
	\end{frame}
	
	%%----------------------------------------------------------------------
	%%----------------------------------------------------------------------
	
	
	\section{Performance Comparison}
	\begin{frame}{Complexity Analysis I}
	For Method I (\textbf{Patchwork Kriging}), the expense is located in the computation of: \vspace{0.1cm}
	\[
	(C_{\delta\delta} - C_{\mathcal{D}\delta}^T C_{\mathcal{D}\mathcal{D}}^{-1} C_{\mathcal{D}\delta})^{-1}
	\]

  	\textbf{Reminder}
    	\begin{itemize}
      		\item<1-> $N$ observations in a space of dimension $d$
 		\item <1-> split into $K$ regions of $M=\frac{N}{K}$ points, with $B$ pseudo-observations at each boundary
   	 \end{itemize}
	

	\pause
 	 \textbf<2->{Complexity}
 	  \begin{itemize}
       		\item<2-> $C_{\mathcal{D}\mathcal{D}}^{-1}$ block diagonal $\rightarrow \boldsymbol{\mathcal{O}(KM^3)}$
       		\item<2-> $C_{\mathcal{D}\mathcal{D}}^{-1} C_{\mathcal{D}\delta}$ $\rightarrow \boldsymbol{\mathcal{O}(KBM^2)}$
       		\item<2-> Inverting $C_{\delta\delta} - C_{\mathcal{D}\delta}^T C_{\mathcal{D}\mathcal{D}}^{-1} C_{\mathcal{D}\delta}$. As it is very sparse, we can use efficient methods to invert sparse matrix (\cite{chan1980linear}) $\longrightarrow \boldsymbol{\mathcal{O}(d^3B^3K)}$
	\end{itemize}
	
	\pause
	$$
	\longrightarrow \boldsymbol{\mathcal{O}(KM^3 +KBM^2 + d^3B^3K )} \approx  \boldsymbol{\mathcal{O}(\frac{N^3}{K^2} + d^3B^3K )}
	$$
	\end{frame}	

	%%----------------------------------------------------------------------
	%%----------------------------------------------------------------------
	

	\begin{frame}{Accuracy}
	For Method I (\textbf{Patchwork Kriging}), the accuracy concerns : \vspace{0.1cm}
    	\begin{itemize}
      		\item <1-> Quality of local reconstruction
 		\item <1-> Quality of "stiches" (junctions of regions)
   	 \end{itemize}
	
	\only<2->{
	\vspace{0.3cm}
	\begin{columns}[T]
 		 \begin{column}{0.5\textwidth}
   			 \begin{figure}
    				\includegraphics[width=.8\textwidth]{figures/Recollement_B_1.png}
    				\vspace*{-0.3cm}
     				\caption{B = 0}
   			 \end{figure}
 		 \end{column}
 		 \begin{column}{0.5\textwidth}
   			 \begin{figure}
    				\includegraphics[width=.8\textwidth]{figures/Recollement_B_2.png}
     				\vspace*{-0.3cm}
      				\caption{B = 5}
   			 \end{figure}
  		\end{column}
	\end{columns}
	}
	\end{frame}	
	
	%%----------------------------------------------------------------------
	%%----------------------------------------------------------------------

	\begin{frame}{Results}
	In \cite{parkchiwoo-2018}, Tests have been made to understand the extent of the Patchwork Kriging Technique

	\begin{itemize}
		\item Result are the most accurate for \textbf{long-range, stationary} covariance functions.
		\item Stitching accuracy is \textbf{conditioned by $B$}, but stops for $B \geq 8$
	\end{itemize}

	\only<2->{
	\vspace{0.3cm}
	\begin{columns}[T]
 		 \begin{column}{0.5\textwidth}
   			 \begin{figure}
    				\includegraphics[width=.8\textwidth]{figures/obs_bruitées.png}
    				\vspace*{-0.3cm}
     				\caption{Observations bruitées}
   			 \end{figure}
 		 \end{column}
 		 \begin{column}{0.5\textwidth}
   			 \begin{figure}
    				\includegraphics[width=.8\textwidth]{figures/champ_prédit.png}
     				\vspace*{-0.3cm}
      				\caption{Champ prédit}
   			 \end{figure}
  		\end{column}
	\end{columns}
	}
	\end{frame}


	%%----------------------------------------------------------------------
	%%----------------------------------------------------------------------

	\begin{frame}{Results}
	In \cite{parkchiwoo-2018}, Tests have been made to understand the extent of the Patchwork Kriging Technique

	\begin{itemize}
		\item Result are the most accurate for \textbf{long-range, stationary} covariance functions.
		\item Stitching accuracy is \textbf{conditioned by $B$}, but stops for $B \geq 8$
	\end{itemize}
	
	\vspace{0.3cm}
	
	\textbf{High dimensionality ?}
	\begin{itemize}
		\item recal that complexity $\longrightarrow \boldsymbol{\mathcal{O}(\frac{N^3}{K^2} + d^3B^3K )}$
		\item experimentally, for $d > 10$, dimension is a determining factor for computation time
	\end{itemize}

	$$
		\rightarrow \text{trade-of quality/speed : keep } \frac{N}{K} \text{ and } dB \text{ as low as possible.}
	$$

	
	\end{frame}	

	%%----------------------------------------------------------------------
	%%----------------------------------------------------------------------
	

	\begin{frame}{Complexity Analysis II}
	For Method I (\textbf{NN-Kriging}), 
	
  	\textbf{Reminder}
    	\begin{itemize}
      		\item <1-> $N$ observations
 		\item <1-> for each new location, a neighbourhood of size $m$ is computed
   	 \end{itemize}

	\pause
 	 \textbf<2->{Complexity}
 	  \begin{itemize}
       		\item<2-> detail ?
	\end{itemize}
	
	\pause
	$$
	\longrightarrow \boldsymbol{\mathcal{O}(Nm^3 )}
	$$
	\end{frame}	
	
	
	%%----------------------------------------------------------------------
	%%----------------------------------------------------------------------

	\section{Conclusion}
	
	\begin{frame}[c]{Conclusion}
		\begin{center}
			Merci pour votre attention !
		\end{center}
			
		\vspace{1.5cm}
		\pause
		\begin{center}
			\begin{quote}
				"On ne peut pas inverser une matrice de taille 20000 une fois, mais on peut inverser 20000 matrice de taille 20 une fois… enfin je crois"\\
			\hfill — \textit{T. Lambert}
			\end{quote}
		\end{center}
	\end{frame}	

	%%----------------------------------------------------------------------
	%%----------------------------------------------------------------------
	
	\appendix
	
	\section*{Appendix}
	
	
	\begin{frame}[allowframebreaks=1]
		\frametitle{References}
		\bibliographystyle{apalike}      % basic style, author-year citations
		\bibliography{biblio}
	\end{frame}
	%
	%
	
	

	
\end{document}